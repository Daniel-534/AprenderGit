\documentclass[a4paper]{article}

\usepackage[spanish]{babel}
\usepackage{graphicx}
\usepackage{amsmath, amssymb}
\usepackage[margin=2cm]{geometry}
\usepackage{fancyhdr}
\usepackage{enumerate}
\usepackage[shortlabels]{enumitem}
\usepackage{parskip}
\usepackage[most]{tcolorbox}
\usepackage[hidelinks]{hyperref}
\usepackage{float}


% cabecera
\pagestyle{fancy}
\fancyhead[l]{Daniel Soto}
\fancyhead[c]{Enlazar Git-GitHub mediante SSH}
\fancyhead[r]{\today}
\fancyfoot[c]{\thepage}
\renewcommand{\headrulewidth}{0.2pt} % linea horizontal


\begin{document}
\textbf{Inicio SSH}

Verificar si se tiene una clave SSH o no.
\begin{verbatim}
cd ~/.ssh
ls -a
\end{verbatim}

Si se observan archivos, borrarlos (si no se necesitan)
\begin{verbatim}
rm *
\end{verbatim}

Generar clave SSH usando el email de la cuenta de GitHub
\begin{verbatim}
ssh-keygen -t ed25519 -C "test@gmail.com"
\end{verbatim}

Agregar nombre, recomendación: \texttt{id\_rsa}
Enter para passphrase (no poner nada)

Continuar con
\begin{verbatim}
eval "($ssh-agent -s)"
\end{verbatim}

Para MacOS es necesario este paso intermedio:

\textbf{Inicio paso intermedio para MacOS}
\begin{verbatim}
touch config
\end{verbatim}

y luego agregar a ese archivo \texttt{config} lo siguiente:
\begin{verbatim}
Host github.com
  AddKeysToAgent yes
  UseKeychain yes
  IdentityFile ~/.ssh/id_rsa
\end{verbatim}

Continuamos con
\begin{verbatim}
ssh-add --apple-use-keychain ~/.ssh/id_rsa
\end{verbatim}

Si la versión de MacOS es anterior a \textit{Monterey (12.0)}, entonces 
no soporta \texttt{--apple-use-keychain}, en cambio, se ejecuta:
\begin{verbatim}
ssh-add -K ~/.ssh/id_rsa
\end{verbatim}
\textbf{Fin paso intermedio para MacOS}

Para Linux:
\begin{verbatim}
ssh-add ~/.ssh/id_rsa
\end{verbatim}

Agregar la clave ubicada en \texttt{~/.ssh/id\_rsa} a GitHub
Settings > SSH and GPG keys > New SSH key

Validar la conexión
\begin{verbatim}
ssh -T git@github.com
.
.
.
yes
\end{verbatim}
\textbf{Inicio SSH}

\textbf{Inicio enlace Git-GitHub}

Supongamos que el usuario de GitHub es \texttt{Daniel-534}
y que el repositorio al que vamos a subir los archivos es \texttt{AprenderGit}.

Crear el repositorio en GitHub. En local, ubicarse en la carpeta donde están los archivos que
se van a subir al repositorio de GitHub y creamos un par de archivos útiles.
\begin{verbatim}
touch README.md
touch .gitignore
\end{verbatim}

Inicializamos el Git y agregamos los archivos de la carpeta.
\begin{verbatim}
git init
git add -
git commit -m "Inicialización"
\end{verbatim}

Renombramos la rama \texttt{master} como \texttt{main}.
\begin{verbatim}
git branch -M main
\end{verbatim}

Apuntamos al repositorio donde queremos subir los archivos
\begin{verbatim}
git remote add origin git@github.com:Daniel-534/AprenderGit.git
\end{verbatim}

En caso de que no funcione, entonces hacer:
\begin{verbatim}
git remote set-url origin git@github.com:Daniel-534/AprenderGit.git
\end{verbatim}

Para el primer \texttt{push}, hacer:
\begin{verbatim}
git push -u origin main
\end{verbatim}

Después, solo hacer:
\begin{verbatim}
git push origin main
\end{verbatim}


\textbf{Fin enlace Git-GitHub}

\end{document}